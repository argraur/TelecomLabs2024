\section{Designing Filter Taps}

Демонстрирует, как создать список или массив коэффициентов фильтра и применить их в блоке низкочастотной фильтрации.

\subsection{Проектирование коэффициентов фильтра}

Начнем с блок-схемы из примера низкочастотного фильтра, но заменим низкочастотный фильтр на частотно-смещенный FIR фильтр и добавим блок Low-Pass Filter Taps:

\begin{center}
    \includegraphics[width=0.8\textwidth]{1}
\end{center}

Блок Low-Pass Filter Taps задает набор коэффициентов фильтра, которые можно применять к блокам фильтрации. Коэффициенты фильтра также могут называться весами или коэффициентами. Ответ и производительность фильтра зависят от параметров, введенных пользователем. Дважды щелкнем по блоку Low-Pass Filter Taps, чтобы открыть свойства. Отредактируем их:

\begin{itemize}
    \item Id: \verb|lowPassFilterTaps|
    \item Cutoff Freq (Hz): \verb|samp_rate/4|
    \item Transition Width (Hz): \verb|samp_rate/8|
\end{itemize}

\begin{center}
    \includegraphics[width=0.5\textwidth]{2}
\end{center}

Блок Low-Pass Filter Taps сохраняет коэффициенты фильтра в список в переменной lowPassFilterTaps.

Дважды щелкнем по блоку частотно-смещенного FIR фильтра, чтобы отредактировать свойства. Введем \verb|lowPassFilterTaps| для параметра Taps. Наведение курсора на переменную lowPassFilterTaps отображает информацию о коэффициентах фильтра:

\begin{center}
    \includegraphics[width=0.8\textwidth]{3}
\end{center}

Первые несколько коэффициентов фильтра отображаются в списке. Сохраним свойства и запустим flowgraph:

\begin{center}
    \includegraphics[width=0.8\textwidth]{4}
\end{center}

Попробуем поизменять частоту. Амплитуда частотной характеристики затем может быть видна через контур:

\begin{center}
    \includegraphics[width=0.8\textwidth]{5}
\end{center}

\subsection{Ввод коэффициентов фильтра вручную}

Альтернативные методы могут использоваться для задания коэффициентов фильтра, а затем ввода их вручную в виде переменной в Python. Например, блок частотно-смещенного FIR фильтра принимает коэффициенты фильтра в виде массива NumPy. Добавим блок Import в рабочее пространство GRC:

\begin{center}
    \includegraphics[width=0.8\textwidth]{6}
\end{center}

\begin{center}
    \includegraphics[width=0.8\textwidth]{7}
\end{center}

Простой фильтр скользящего среднего или \verb|boxcar| можно задать, установив все коэффициенты фильтра одинаковыми. Это можно сделать с помощью функции NumPy ones(), которая возвращает массив NumPy, состоящий из единиц указанной длины. Создадим переменную с именем boxcarFilter со значением:

\begin{verbatim}
boxcarFilter = np.ones(10)
\end{verbatim}

\begin{center}
    \includegraphics[width=0.8\textwidth]{8}
\end{center}

Отключим Low-Pass Filter Taps и заменим его на \verb|boxcarFilter| в Frequency Xlating FIR Filter. Блок-схема выглядит следующим образом:

\begin{center}
    \includegraphics[width=0.8\textwidth]{9}
\end{center}

Теперь можно увидеть другую амплитуду частотной характеристики, поскольку используются разные коэффициенты фильтра:

\begin{center}
    \includegraphics[width=0.8\textwidth]{10}
\end{center}

\subsection{Фильтр от реального к комплексному}

Многие блоки фильтрации имеют опции выбора комбинаций реальных или комплексных типов данных для входа и выхода, а также реальных или комплексных весов фильтра. Этот пример демонстрирует один из методов использования комплексных весов фильтра для преобразования реального сигнала в комплексный. Удалим переменную \verb|boxcarFilter| и снова включим блок Low-Pass Filter Taps:

\begin{center}
    \includegraphics[width=0.8\textwidth]{11}
\end{center}

\verb|lowPassFilterTaps| используются как основа для комплексного полосового фильтра. Создадим переменную \verb|n| со значением:

\begin{verbatim}
np.arange(0,len(lowPassFilterTaps))
\end{verbatim}

которая образует массив целых чисел: 0, 1, 2, 3, ... до длины \verb|lowPassFilterTaps|:

\begin{center}
    \includegraphics[width=0.8\textwidth]{12}
\end{center}

Создадим переменную \verb|frequencyShift| со значением:

\begin{verbatim}
np.exp(1j * 2 * np.pi * 0.25 * n)
\end{verbatim}

\begin{center}
    \includegraphics[width=0.8\textwidth]{13}
\end{center}

Эта функция является комплексной синусоидой с частотой \(1/4\) частоты дискретизации. Переменная \verb|frequencyShift| изменяет центральную частоту \verb|lowPassFilterTaps| с \(0\) до \(1/4\) частоты дискретизации. Создадим переменную \verb|bandPassTaps| со значением:

\begin{verbatim}
lowPassFilterTaps * frequencyShift
\end{verbatim}

\begin{center}
    \includegraphics[width=0.8\textwidth]{14}
\end{center}

Отредактируем блок Frequency Xlating FIR Filter. В качестве типа выберем Float->Complex(Complex Taps):

\begin{center}
    \includegraphics[width=0.8\textwidth]{15}
\end{center}

Заменим lowPassFilterTaps на bandPassTaps в частотно-смещенном фильтре.

Отредактируем свойства источника сигнала и преобразуем его в реальный сигнал.

\begin{center}
    \includegraphics[width=0.8\textwidth]{16}
\end{center}

Блок-схема выглядит следующим образом:

\begin{center}
    \includegraphics[width=0.8\textwidth]{17}
\end{center}

Запустим блок-схему.

\begin{center}
    \includegraphics[width=0.8\textwidth]{18}
\end{center}

Амплитуда частотной характеристики показывает, что центральная частота низкочастотного фильтра была смещена до 1/4 частоты дискретизации, что теперь является полосовым фильтром. Частотная характеристика теперь отличается между положительными и отрицательными частотами, что может быть свойством комплексных фильтров (но не реальных фильтров).